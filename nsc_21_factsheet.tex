% Options for packages loaded elsewhere
\PassOptionsToPackage{unicode}{hyperref}
\PassOptionsToPackage{hyphens}{url}
\PassOptionsToPackage{dvipsnames,svgnames,x11names}{xcolor}
%
\documentclass[
  11pt,
]{article}
\author{}
\date{\vspace{-2.5em}}

\usepackage{amsmath,amssymb}
\usepackage{lmodern}
\usepackage{iftex}
\ifPDFTeX
  \usepackage[T1]{fontenc}
  \usepackage[utf8]{inputenc}
  \usepackage{textcomp} % provide euro and other symbols
\else % if luatex or xetex
  \usepackage{unicode-math}
  \defaultfontfeatures{Scale=MatchLowercase}
  \defaultfontfeatures[\rmfamily]{Ligatures=TeX,Scale=1}
  \setsansfont[]{Roboto}
\fi
% Use upquote if available, for straight quotes in verbatim environments
\IfFileExists{upquote.sty}{\usepackage{upquote}}{}
\IfFileExists{microtype.sty}{% use microtype if available
  \usepackage[]{microtype}
  \UseMicrotypeSet[protrusion]{basicmath} % disable protrusion for tt fonts
}{}
\makeatletter
\@ifundefined{KOMAClassName}{% if non-KOMA class
  \IfFileExists{parskip.sty}{%
    \usepackage{parskip}
  }{% else
    \setlength{\parindent}{0pt}
    \setlength{\parskip}{6pt plus 2pt minus 1pt}}
}{% if KOMA class
  \KOMAoptions{parskip=half}}
\makeatother
\usepackage{xcolor}
\IfFileExists{xurl.sty}{\usepackage{xurl}}{} % add URL line breaks if available
\IfFileExists{bookmark.sty}{\usepackage{bookmark}}{\usepackage{hyperref}}
\hypersetup{
  colorlinks=true,
  linkcolor={Maroon},
  filecolor={Maroon},
  citecolor={Blue},
  urlcolor={Blue},
  pdfcreator={LaTeX via pandoc}}
\urlstyle{same} % disable monospaced font for URLs
\usepackage[left=1in,right=1in,top=0.35in,bottom=0.6in]{geometry}
\usepackage{graphicx}
\makeatletter
\def\maxwidth{\ifdim\Gin@nat@width>\linewidth\linewidth\else\Gin@nat@width\fi}
\def\maxheight{\ifdim\Gin@nat@height>\textheight\textheight\else\Gin@nat@height\fi}
\makeatother
% Scale images if necessary, so that they will not overflow the page
% margins by default, and it is still possible to overwrite the defaults
% using explicit options in \includegraphics[width, height, ...]{}
\setkeys{Gin}{width=\maxwidth,height=\maxheight,keepaspectratio}
% Set default figure placement to htbp
\makeatletter
\def\fps@figure{htbp}
\makeatother
\setlength{\emergencystretch}{3em} % prevent overfull lines
\providecommand{\tightlist}{%
  \setlength{\itemsep}{0pt}\setlength{\parskip}{0pt}}
\setcounter{secnumdepth}{-\maxdimen} % remove section numbering
\input{preamble.tex}
\ifLuaTeX
  \usepackage{selnolig}  % disable illegal ligatures
\fi

\begin{document}

\raggedright

\sdcoelogo{}

\vspace{20mm}
\sdcoetitle{Declines in College Enrollment Continue Unabated in 2021}
\sdcoeauthors{Shannon Coulter, Director of Research and Evaluation}

\vspace{5mm}

\lettrine[lines=3, lraise=0.2, nindent=0em, slope=0em]{H}igher education
enrollment fell to its lowest levels in San Diego in 2021. In San Diego
County, an estimated 5,250 fewer students opted in to college compared
to five years ago. Such a decline poses a serious risk for many
students, especially those students who rely on a college degree to
become upwardly mobile. And yet, simply enrolling in college is no
panacea to upward mobility, students must actually attain a college
degree, which is no small feat given that fewer than 25 percent actually
do. Because of these historic declines in enrollment, we face a
monumental problem: We currently have fewer students in the pipeline to
attain a college degree--potentially losing 1,000 degree earners from
the Class of 2021 alone--and this problem has the potential to further
erode a fundamental purpose of K-12 education, our higher education
institutions, and our economy at large.

Decisions about attending college were challenging prior to the
pandemic. Students often questioned their readiness for the rigor of
college work and the financial costs. The pandemic has layered other
concerns that make college-going decisions even more unstable for
students.

There is no doubt that many students need a college degree to attain
their professional aspirations and to achieve economic mobility. When
someone's income improves over their lifetime that person is considered
upwardly mobile, so those students completing a four-year degree have a
better chance of becoming middle class than students who don't.
Additionally, San Diego needs degree earners because we need to keep
high-quality talent in the workforce in order to grow our economy. These
problems are complex and we need to act on them in two ways. First, in
the short-term, we must improve our understanding of college-going data
through SDCOE's partnership with high school districts and the National
Student Clearinghouse. That information will help educators address
students' fears and hesitations about attaining a college degree and
boost college access. Second, we must implement long-term solutions such
as improving college and career readiness in the K-12 system and
supporting lower income and disadvantaged students to complete college
degrees.

The annual report of the National Clearinghouse data is designed to
increase our understanding and decisions about college access and
completion. The National Student Clearinghouse's StudentTracker service
is a nationwide source of college enrollment and degree data from over
3,600 colleges and universities. The StudentTracker data provides our
community with a reasonably accurate representation of the number and
percentage of students who enroll, persist, and complete college
annually. The SDCOE partnership with local high school districts to
capture college-going data guide our efforts to support more students in
attending and completing a four-year degree so they and our region can
thrive.

\contactinfo{}

\newpage{}
\newgeometry{left=1in,right=1in,top=1in,bottom=1.5in}

\sdcoeheadingone{College Enrollment}

In 2021, the National Student Clearinghouse reported 18,915 San Diego
County students enrolled in college right out of high school. College
enrollment patterns in the Clearinghouse may vary from year to year for
a variety of reasons including an overall decline in college going,
fluctuations in the size of 12th grade cohorts, and the timing of when
local high schools submit their graduates to the Clearinghouse. For that
reason, this report is a preliminary report that is updated in late
spring.

The map below locates the nearly 1,900 colleges and universities where
students enrolled in 2021. Most students (83\%) attended colleges and
universities in Southern California, while the remaining 2,200 attended
institutions in all other 49 states.

\sdcoefigurenumber{1}
\sdcoefiguretitle{While Many College Students Stay Close to Home, Graduates Enroll in Colleges and Universities Across the Country}

\includegraphics{nsc_21_factsheet_files/figure-latex/enroll-map-1.pdf}
\sdcoesource{National Student Clearinghouse StudentTracker Report}
\sdcoenote{We used geospatial data to identify the location of each college or university where graduates from the San Diego County Class of 2021 enrolled. We mapped each student to their respective college or university, darker dot patterns represent areas where more students attend college.}

\newpage{}
\newgeometry{left=1in,right=1in,top=1in,bottom=1.5in}

\sdcoeheadingtwo{Top 10 Colleges and Universities Attended}

For the first time in eight years more students enrolled in four-year
institutions (9,633) compared to two-year (8,007). Some local community
colleges suffered enrollment losses upward of 25\% in 2021, while UC-San
Diego reported a 26\% increase in enrollment compared to the past two
years. These variations in enrollment patterns have the greatest impact
on lower-income students whose enrollment declines were double other
student groups, according to the National Student Clearinghouse.

\vspace{3mm}
\sdcoetablenumber{1}
\sdcoetabletitle{More High School Graduates Opted for a Four-Year College Experience in 2021}
\sdcoetablesubtitle{Community colleges still have a big enrollment footprint in San Diego}

\begin{tabu} to \linewidth {>{\raggedright\arraybackslash}p{7cm}>{\raggedleft\arraybackslash}p{4.75cm}>{\raggedleft}X}
\toprule
\multicolumn{1}{c}{ } & \multicolumn{1}{c}{ } & \multicolumn{1}{c}{ } \\

\textbf{ } & \textbf{Level} & \textbf{Number Attending}\\
\midrule
\addlinespace[0.3em]
\multicolumn{3}{l}{\textbf{College Name}}\\
\hspace{1em}San Diego State University & 4-year & 1501\\
\hspace{1em}University of California-San Diego & 4-year & 1310\\
\hspace{1em}California State University-San Marcos & 4-year & 959\\
\hspace{1em}University of California-Los Angeles & 4-year & 298\\
\hspace{1em}Southwestern College & 2-year & 2004\\
\hspace{1em}Palomar College & 2-year & 1827\\
\hspace{1em}San Diego Mesa College & 2-year & 1073\\
\hspace{1em}Grossmont College & 2-year & 953\\
\hspace{1em}San Diego Miramar College & 2-year & 874\\
\hspace{1em}San Diego City College & 2-year & 711\\
\bottomrule
\end{tabu}
\sdcoesource{National Student Clearinghouse StudentTracker Report}

\newpage{}
\newgeometry{left=1in,right=1in,top=1in,bottom=1.5in}

\sdcoeheadingtwo{College Enrollment by Cohort}

The overall number of students enrolled in college was down in 2021
compared to enrollment patterns prior to the pandemic. As indicated in
the chart below, the number of students opting out of college (dark
blue) increases over time, while the number enrolled (light blue)
decreases over the same time period. Basically, a substantial number of
students are opting out of college today compared to the past seven
years.

\sdcoefigurenumber{2}
\sdcoefiguretitle{Early Enrollment Figures Show Another Steep Decline in College-Going}
\vspace{3mm}

\includegraphics{nsc_21_factsheet_files/figure-latex/enroll-by-cohort-1.pdf}
\sdcoesource{National Student Clearinghouse StudentTracker Report}
\sdcoenote{We calculated enrollment number for each graduating cohort using data provided by the National Student Clearinghouse. The Clearinghouse data file supplies information on whether a student enrolled in college or not with a 93 percent matching rate.}

\newpage{}
\newgeometry{left=1in,right=1in,top=1in,bottom=1.5in}

\sdcoeheadingone{Persistence and Retention}

Despite the pandemic, students who enroll in college tend to stay. Some
stay at their original institution (retention) while others opt to
attend college elsewhere (persistence). While persistence and retention
percentages declined slightly for the Class of 2020, differences between
persistence and retention rates remained steady. About 4 in 5 enrolled
students remained enrolled from year to year.

\sdcoefigurenumber{3}
\sdcoefiguretitle{Differences in College Persistence and Retention Hold Steady}
\sdcoefiguresubtitle{The number of students that remain in college compared to those staying in the same college}
\vspace{3mm}

\includegraphics{nsc_21_factsheet_files/figure-latex/persist-by-cohort-1.pdf}
\sdcoesource{National Student Clearinghouse StudentTracker Report}
\sdcoenote{We calculated persistence as the percentage of students enrolled in college in the first year after high school graduation who then returned to any college for a second year. We calculated retention as the percentage of students returning to the same college from the previous year.}

\newpage{}
\newgeometry{left=1in,right=1in,top=1in,bottom=1.5in}

\sdcoeheadingone{Completion}

While the pandemic has made no discernible difference on college
completion rates yet, historically only a small number of students
actually earn college degrees. About 19\% of the Class of 2017 has
earned a college degree, while 28\% of the Class of 2015 earned a
degree. To improve economic mobility, supporting students to access
college isn't enough. We need to get students to finish. Furthermore, we
might have significantly fewer students earning degrees in the future
based on our current enrollment levels.

\sdcoefigurenumber{4}
\sdcoefiguretitle{More than 1 in 4 San Diego Students Earn a College Degree in 6 Years}
\sdcoefiguresubtitle{The percentange of students completing degrees in four and six years remains low}

\includegraphics{nsc_21_factsheet_files/figure-latex/completion-by-cohort-1.pdf}
\sdcoesource{National Student Clearinghouse StudentTracker Report}
\sdcoenote{Colleges report graduation rates many different ways. We calculated graduation rates by cohort or year of high school graduation. The six-year graduation rate reflected on the graph represents the percentage of students from the Class of 2015 who earned a degree. The four-year rate represents only the students from the class of 2017 who have earned a degree.}

\newpage{}
\newgeometry{left=1in,right=1in,top=1in,bottom=1.5in}

\sdcoeheadingtwo{Degrees Earned}

For students who persisted and earned a degree, Business Administration
topped the list as the most popular college major followed by Psychology
and Computer Science. Because earnings vary based on the type of degree
students get, several organizations including
\href{https://collegescorecard.ed.gov/}{College Scorecard} and
\href{https://salarysurfer.cccco.edu/SalarySurfer.aspx}{Salary Surfer}
provide valuable information linking college degrees to the median
salaries of graduates. Students may be surprised to learn that some
associate's degrees may result in higher earning potential than some
bachelor's degrees, which could reduce worries or hesitation about
enrolling in a two- or four-year schools.

\sdcoetablenumber{2}
\sdcoetabletitle{2014 Graduates Earned Nearly 7,500 Degrees}
\sdcoetablesubtitle{Many students earn two-year degrees in same time some students earn four-year degrees}

\begin{tabu} to \linewidth {>{\raggedright\arraybackslash}p{7cm}>{\raggedleft\arraybackslash}p{4.75cm}>{\raggedright}X}
\toprule
\multicolumn{1}{c}{ } & \multicolumn{1}{c}{ } & \multicolumn{1}{c}{ } \\

\textbf{ } & \textbf{\# Obtained} & \textbf{Time to Completion}\\
\midrule
\addlinespace[0.3em]
\multicolumn{3}{l}{\textbf{Degree}}\\
\hspace{1em}Certificate & 131 & 3 years, 11 months\\
\hspace{1em}Associate's & 347 & 4 years 1 month\\
\hspace{1em}Bachelor's & 3082 & 4 years 6 months\\
\hspace{1em}Master's and beyond (with Bachelor's) & 352 & 5 years 11 months\\
\hspace{1em}Unreported & 3418 & 4 years 6 months\\
\addlinespace[0.3em]
\multicolumn{3}{l}{\textbf{Majors}}\\
\hspace{1em}Business Administration & 722 & \\
\hspace{1em}Psychology & 503 & \\
\hspace{1em}Computer Science & 424 & \\
\hspace{1em}Political Science & 256 & \\
\hspace{1em}Economics & 254 & \\
\hspace{1em}Sociology & 250 & \\
\hspace{1em}Communication & 229 & \\
\hspace{1em}Mechanical Engineering & 228 & \\
\hspace{1em}Biology & 208 & \\
\bottomrule
\end{tabu}
\sdcoesource{National Student Clearinghouse StudentTracker Report}
\sdcoenote{We calculated degree information and time to degree from the National Student Clearinghouse StudentTracker report using the Class of 2014. Many colleges and universities choose not to report degree titles or degree majors opting instead to report a graduation date only. For those students, we have identified their degree as "unreported."}

\newpage{}
\newgeometry{left=1in,right=1in,top=1in,bottom=1.5in}

\sdcoeheadingone{Final Thoughts}

Imagine you are one of the 40,648 graduating seniors from one of San
Diego's 175 high schools. Going to college may have been an aspiration
of yours for as long as you can remember. How do you feel right now?
Perhaps you were already worried about how to pay for college, making
friends, or failing a class. Now you may also be anxious about how the
pandemic has changed your college dreams. What would it take to allay
your fears and get you back on track for postsecondary options?

The fact is that Covid-19 or any of its variants will have an impact on
college-going for some time. We cannot end the pandemic as a solution to
improving college access. The best chance we have of ending this
enrollment crisis is to build stronger relationships with students and
to provide them with information about their college-going options that
keeps them progressing. Additionally, we need to better understand the
social and emotional stressors they experience in the process and work
to mitigate these challenges. Going to college is not something we can
leave to chance.

Everyone reading this report is an important decision-maker in their
organization---many are directly responsible for college access. I urge
you to stay informed about shifts in college enrollment patterns and
invest in strategies that address students' fears and hesitations about
enrolling in college. This group of high school graduates may be the
most resilient of any other cohort before them. We must honor their
experiences because they have truly paid a much greater price to achieve
their aspirations than any of us ever have or possibly will.

\vspace{10mm}
\begin{tcolorbox}[colback=sdcoelightblue!5, colframe=sdcoelightblue!75]
\sdcoeheadingone{Interested in boosting four-year college enrollment rates?}

Check out these resources:

\begin{sdcoebullets}
  \item \href{https://resources.finalsite.net/images/v1633543351/sdcoenet/n9ck4tbjbsexckrvxrgb/CollegeAccess-Brochure.pdf}{Is a four-year college right for me?}
  \item \href{https://resources.finalsite.net/images/v1639429972/sdcoenet/kmq4ljy195amwtiinc4w/MakeItEasier-Report.pdf}{Make it Easier}
  \item \href{https://www.sdcoe.net/educators/evaluation/improvement-networks#fs-panel-2947}{College Enrollment Checklists}
\end{sdcoebullets}
\end{tcolorbox}

\sdcoeboilerplate{SDCOE's Assessment, Accountability, and Evaluation Department}{February}{2022}

\end{document}
