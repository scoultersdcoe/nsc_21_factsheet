% Options for packages loaded elsewhere
\PassOptionsToPackage{unicode}{hyperref}
\PassOptionsToPackage{hyphens}{url}
\PassOptionsToPackage{dvipsnames,svgnames,x11names}{xcolor}
%
\documentclass[
  11pt,
]{article}
\author{}
\date{\vspace{-2.5em}}

\usepackage{amsmath,amssymb}
\usepackage{lmodern}
\usepackage{iftex}
\ifPDFTeX
  \usepackage[T1]{fontenc}
  \usepackage[utf8]{inputenc}
  \usepackage{textcomp} % provide euro and other symbols
\else % if luatex or xetex
  \usepackage{unicode-math}
  \defaultfontfeatures{Scale=MatchLowercase}
  \defaultfontfeatures[\rmfamily]{Ligatures=TeX,Scale=1}
  \setsansfont[]{Roboto}
\fi
% Use upquote if available, for straight quotes in verbatim environments
\IfFileExists{upquote.sty}{\usepackage{upquote}}{}
\IfFileExists{microtype.sty}{% use microtype if available
  \usepackage[]{microtype}
  \UseMicrotypeSet[protrusion]{basicmath} % disable protrusion for tt fonts
}{}
\makeatletter
\@ifundefined{KOMAClassName}{% if non-KOMA class
  \IfFileExists{parskip.sty}{%
    \usepackage{parskip}
  }{% else
    \setlength{\parindent}{0pt}
    \setlength{\parskip}{6pt plus 2pt minus 1pt}}
}{% if KOMA class
  \KOMAoptions{parskip=half}}
\makeatother
\usepackage{xcolor}
\IfFileExists{xurl.sty}{\usepackage{xurl}}{} % add URL line breaks if available
\IfFileExists{bookmark.sty}{\usepackage{bookmark}}{\usepackage{hyperref}}
\hypersetup{
  colorlinks=true,
  linkcolor={Maroon},
  filecolor={Maroon},
  citecolor={Blue},
  urlcolor={Blue},
  pdfcreator={LaTeX via pandoc}}
\urlstyle{same} % disable monospaced font for URLs
\usepackage[left=1in,right=1in,top=0.35in,bottom=0.6in]{geometry}
\usepackage{graphicx}
\makeatletter
\def\maxwidth{\ifdim\Gin@nat@width>\linewidth\linewidth\else\Gin@nat@width\fi}
\def\maxheight{\ifdim\Gin@nat@height>\textheight\textheight\else\Gin@nat@height\fi}
\makeatother
% Scale images if necessary, so that they will not overflow the page
% margins by default, and it is still possible to overwrite the defaults
% using explicit options in \includegraphics[width, height, ...]{}
\setkeys{Gin}{width=\maxwidth,height=\maxheight,keepaspectratio}
% Set default figure placement to htbp
\makeatletter
\def\fps@figure{htbp}
\makeatother
\setlength{\emergencystretch}{3em} % prevent overfull lines
\providecommand{\tightlist}{%
  \setlength{\itemsep}{0pt}\setlength{\parskip}{0pt}}
\setcounter{secnumdepth}{-\maxdimen} % remove section numbering
\input{preamble.tex}
\ifLuaTeX
  \usepackage{selnolig}  % disable illegal ligatures
\fi

\begin{document}

\raggedright

\sdcoelogo{}

\vspace{5mm}
\sdcoetitle{Declines in College Enrollment Continue Unabated in 2021}
\sdcoeauthors{Shannon Coulter, Director of Research and Evaluation}

\vspace{5mm}
\begin{adjustwidth}{50pt}{50pt}
\lettrine[lines=3, lraise=0.2, nindent=0em, slope=0em]{H}igher education enrollment fell to its lowest levels in 2021. In San Diego, more than 5,250 fewer students opted in to college compared to five years ago. Such a decline poses a serious risk for many students, especially those students who rely on a college degree to become upwardly mobile. And yet, simply enrolling in college is no panacea to upward mobility, students must actually attain a college degree, which is no small feat given that fewer than 25 percent actually do. Because of these historic declines in enrollment, we face a monumental problem: We currently have fewer students in the pipeline to attain a college degree--potentially losing 1,000 degree earners from the Class of 2021 alone--and this problem has the potential to further erode a fundamental purpose of K-12 education, our higher education institutions, and our economy at large.

Decisions about attending college were challenging prior to the pandemic. Students often questioned their readiness for the rigor of college work and the financial costs. The pandemic has layered other concerns that make college-going decisions even more unstable for students. 

There is no doubt that many students need a college degree to attain their professional aspirations and to achieve economic mobility. San Diego also needs degree earners--the relationship between economic growth and educational attainment are undeniable. We need to act on these complex problems now and address them with long terms solutions, and in the short term, we must continue to improve our understanding of our college-going data and use this information to mitigate students' fears and hesitations about attaining a college degree.

We created this annual report for the purpose of increasing our understanding and decisions about college access and completion. Instead of relying on self-reported information about how many students go to college,  we use the National Student Clearinghouse's StudentTracker service, a nationwide source of college enrollment and degree data from over 3,600 colleges and universities. The StudentTracker data provides our community with a reasonably accurate representation of the number and percentage of students who enroll, persist, and complete college annually. This report helps guide our efforts to get our students back into college and keep our economic outlook bright.
\end{adjustwidth}

\contactinfo{}

\newpage{}
\newgeometry{left=1in,right=1in,top=1in,bottom=1.5in}

\sdcoeheadingone{College Enrollment}

In 2021, the National Student Clearinghouse reported 18,915 students
enrolled in college right out of high school, a slight decline from the
year before. College enrollment patterns in the Clearinghouse may vary
from year to year for a variety of reasons including an overall decline
in college going, fluctuations in the size of 12th grade cohorts, and
the timing of when local high schools submit their graduates to the
Clearinghouse. For that reason, our fall report is a preliminary report
that is updated in late spring.

The map below locates the nearly 1,900 colleges and universities where
students enrolled in 2021. Most students (83\%) attended colleges and
universities in southern California, while the remaining 2,200 attended
institutions in all other 49 states.

\sdcoefigurenumber{1}
\sdcoefiguretitle{While Most College Students Stay Close to Home, Graduates Enroll in Colleges and Universities Across the Country}

\includegraphics{gothold_talking_points_2_files/figure-latex/enroll-map-1.pdf}
\sdcoesource{National Student Clearinghouse StudentTracker Report.}
\sdcoenote{We used geospatial data to identify the location of each college or university where graduates from the Class of 2021 enrolled. We mapped each student to his or her respective college or university, darker dot patterns represent areas where more students attend college.}

\newpage{}
\newgeometry{left=1in,right=1in,top=1in,bottom=1.5in}

\sdcoeheadingtwo{Top Ten Colleges and Universities Attended}

For the first time in eight years more students enrolled in four-year
institutions (9,633) compared to 2-year (8,007). Some local community
colleges suffered enrollment loses upwards of 25\% in 2021, while the
University of California-San Diego reported a 26\% increase in
enrollment compared to the past two years. These variations in
enrollment patterns have the greatest impact on lower-income students
whose enrollment declines were double other student groups according to
the National Student Clearinghouse.

\vspace{3mm}
\sdcoetablenumber{1}
\sdcoetabletitle{More High School Graduates Opted for a Four-Year College Experience in 2021}
\sdcoetablesubtitle{Community colleges still have a big enrollment footprint in San Diego}

\begin{tabu} to \linewidth {>{\raggedright\arraybackslash}p{7cm}>{\raggedleft\arraybackslash}p{4.75cm}>{\raggedleft}X>{\raggedleft}X}
\toprule
\multicolumn{1}{c}{ } & \multicolumn{1}{c}{ } & \multicolumn{1}{c}{ } & \multicolumn{1}{c}{ } & \multicolumn{1}{c}{ } \\

\textbf{ } & \textbf{Freshman Class} & \textbf{Pell grad share} & \textbf{Net price, middle-income}\\
\midrule
\addlinespace[0.3em]
\multicolumn{4}{l}{\textbf{College Name}}\\
\hspace{1em}University of California-Irvine & 5449 & 40 & \$13k\\
\hspace{1em}University of California-Davis & 5063 & 31 & \$14k\\
\hspace{1em}University of California-Santa Barbara & 4597 & 31 & \$14k\\
\hspace{1em}University of California-San Diego & 5218 & 28 & \$13k\\
\hspace{1em}University of California-Los Angeles & 5684 & 28 & \vphantom{1} \$13k\\
\hspace{1em}University of California-Los Angeles & 5684 & 28 & \$13k\\
\hspace{1em}University of Florida & 6348 & 24 & \$9k\\
\hspace{1em}University of California-Berkeley & 4677 & 23 & \$13k\\
\hspace{1em}Vassar & 662 & 22 & \$12k\\
\hspace{1em}Amherst & 466 & 20 & \$11k\\
Pomona & 396 & 18 & \$9k\\
\bottomrule
\end{tabu}
\sdcoesource{National Student Clearinghouse StudentTracker Report.}

\newpage{}
\newgeometry{left=1in,right=1in,top=1in,bottom=1.5in}

\sdcoeheadingtwo{College Enrollment by Cohort}

San Diego's total undergraduate enrollment has declined over 3200
students in the past 2 years. The loss to these students' futures is
significant and will greatly impact our community in years to come. We
need to re-engage these and all students in the value of a college
education.

\sdcoefigurenumber{2}
\sdcoefiguretitle{Early Enrollment Figures Show Another Steep Decline in College-Going}
\vspace{3mm}

\includegraphics{gothold_talking_points_2_files/figure-latex/enroll-by-cohort-1.pdf}
\sdcoesource{National Student Clearinghouse StudentTracker Report.}
\sdcoenote{We calculated enrollment numbers for each graduating cohort using data provided by the National Student Clearinghouse. The Clearinghouse data file supplies informatin on whether a student enrolled in college or not with a 93 percent matching rate.}

\newpage{}
\newgeometry{left=1in,right=1in,top=1in,bottom=1.5in}

\sdcoeheadingone{Completion}

While the pandemic has made no discernible difference on college
completion rates yet, only a small number of students actually earn
college degrees. About 17\% of the Class of 2017 has earned a college
degree, while 24\% of the Class of 2015 earned a degree. We know that
degree completers earn more on average than non-college graduates, and
that we need a certain level of degree completers for our economy to
thrive. While the percentage of students earning degrees compared to
those enrolled may not change much in the coming years, we might have
significantly fewer students earning degrees in the future based on our
current enrollment levels.

\sdcoefigurenumber{4}
\sdcoefiguretitle{Only 1 in 4 San Diego Students Earn a College Degree}
\sdcoefiguresubtitle{The percentange of students completing degrees in four and six years remains low}

\includegraphics{gothold_talking_points_2_files/figure-latex/completion-by-cohort-1.pdf}
\sdcoesource{National Student Clearinghouse StudentTracker Report.}
\sdcoenote{Colleges report graduation rates many different ways. We calculated graduation rates by cohort or year of high school graduation. The 2015 cohort represents the 6-year graduation cohort and the data point reflects the percentage of students from that cohort who earned a degree.}

\newpage{}
\newgeometry{left=1in,right=1in,top=1in,bottom=1.5in}

\sdcoeheadingtwo{Degrees Earned}

Business Administration degrees top the list as the most popular college
choice followed by Psychology and Computer Science. Because earnings
vary based on the type of degree students get, several organizations
including \href{https://collegescorecard.ed.gov/}{College Scorecard} and
\href{https://salarysurfer.cccco.edu/SalarySurfer.aspx}{Salary Surfer}
provide valuable information linking college degrees to the median
salaries of graduates. Surprisingly the average salary for some
Associate's degrees are as much if not more than comparable Bachelor's
degrees. Using earnings information can help reduce some students' fears
and hesitations about enrolling in college.

\sdcoetablenumber{2}
\sdcoetabletitle{2014 Graduates Earned Over 7500 Degrees}
\sdcoetablesubtitle{Many students earn two-year degrees in same time some students earn four-year degrees}

\begin{tabu} to \linewidth {>{\raggedright\arraybackslash}p{7cm}>{\raggedleft\arraybackslash}p{4.75cm}>{\raggedright}X}
\toprule
\multicolumn{1}{c}{ } & \multicolumn{1}{c}{ } & \multicolumn{1}{c}{ } \\

\textbf{ } & \textbf{\# Obtained} & \textbf{Time to Completion}\\
\midrule
\addlinespace[0.3em]
\multicolumn{3}{l}{\textbf{Degree}}\\
\hspace{1em}Certificate & 313 & 3 years 11 months\\
\hspace{1em}Associate's & 347 & 4 years 1 month\\
\hspace{1em}Bachelor's & 3082 & 4 years 6 months\\
\hspace{1em}Master's and beyond (with Bachelors) & 352 & 5 years 11 months\\
\hspace{1em}Unreported & 3418 & 4 years 6 months\\
\addlinespace[0.3em]
\multicolumn{3}{l}{\textbf{Majors}}\\
\hspace{1em}Business Administration & 722 & \\
\hspace{1em}Psychology & 503 & \\
\hspace{1em}Computer Science & 424 & \\
\hspace{1em}Political Science & 256 & \\
\hspace{1em}Economics & 254 & \\
\hspace{1em}Sociology & 250 & \\
\hspace{1em}Communication & 229 & \\
\hspace{1em}Mechanical Engineering & 228 & \\
\hspace{1em}Biology & 208 & \\
\bottomrule
\end{tabu}
\sdcoesource{National Student Clearinghouse StudentTracker Report.}
\sdcoenote{We calculated degree information and time to degree from the National Student Clearinghouse StudentTracker report using the Class of 2014. Many colleges and universities choose not to report degree titles or degree majors opting instead to report a graduation date only. For those students, we have identified their degree as "unreported."}

\newpage{}
\newgeometry{left=1in,right=1in,top=1in,bottom=1.5in}

\sdcoeheadingone{Final Thoughts}

Imagine you are one of the 40,648 graduating seniors from one of San
Diego's 175 high schools. Going to college may have been an aspiration
of yours for as long as you can remember. How do you feel right now?
Perhaps you were already worried about how to pay for college, making
friends, or failing a class. Now you may also be anxious about how the
pandemic has changed your college dreams. What would it take to allay
your fears and get you back on track for postsecondary options?

The fact is that Omicron or Covid-19 will have an impact on
college-going for some time. We cannot end the pandemic as a solution to
improving college access. The best chance we have of ending this
enrollment crisis is to build stronger relationships with students and
to provide them with information about their college-going options that
is informative and keeps them progressing in the process. Additionally,
we need to continue to understand the social and emotional stressors
they experience in the process and work to mitigate these challenges.
Going to college is not something we can leave to chance.

Everyone reading this report is an important decision-maker in his or
her organization---many are directly responsible for college access. I
urge you to stay informed about shifts in college enrollment patterns
and invest in strategies that address students' fears and hesitations
about enrolling in college. This group of high school graduates may be
the most resilient of any other cohort before them. We must honor their
experiences because they have truly paid a much greater price to achieve
their aspirations than any of us every have or possibly will.

\sdcoeboilerplate{SDCOE's Assessment, Accountability, and Evaluation Department}{January}{2022}

\end{document}
